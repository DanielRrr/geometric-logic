\documentclass[a4paper]{article}
\usepackage{amsmath}
\usepackage{amsthm}
\usepackage{amsfonts}
\usepackage{amssymb}
\usepackage{bussproofs}
\usepackage{mathtools}
\usepackage{verbatim}
\usepackage{dsfont}
\usepackage{mathabx}
\usepackage[all, 2cell]{xy}
\usepackage[all]{xy}
\usepackage{wasysym}
\usepackage{rotating}
\usepackage{geometry}
\usepackage{trfsigns}
\usepackage{cmll}
\usepackage{authblk}
\usepackage{hyperref}
\usepackage{cleveref}
\usepackage{lipsum}
\usepackage{extpfeil}
\usepackage{soul}
\usepackage{graphicx}

\newcommand\mapsfrom{\mathrel{\reflectbox{\ensuremath{\mapsto}}}}

\theoremstyle{defin}
\newtheorem{definition}{Definition}

\theoremstyle{theorem}
\newtheorem{theorem}{Theorem}

\theoremstyle{claim}
\newtheorem{claim}{Claim}

\theoremstyle{prop}
\newtheorem{prop}{Proposition}

\theoremstyle{lemma}
\newtheorem{lemma}{Lemma}

\theoremstyle{fact}
\newtheorem{fact}{Fact}

\theoremstyle{ex}
\newtheorem{ex}{Example}


\theoremstyle{col}
\newtheorem{col}{Corollary}

\let\strokeL\L
\DeclareRobustCommand{\L}{\ifmmode\mathbf{L}\else\strokeL\fi}

\author{Daniel Rogozin}
\date{}
\title{Notes on Geometric logic}

\begin{document}

\maketitle

\section{Sheaves, Sites and Grothendieck toposes}

Let $\mathcal{I} = (I, \theta)$ be a topological space. Consider $\theta$ as a poset. A {\emph presheaf} over $\mathcal{I}$ is a contravariant functor from $\theta$ to ${\bf Set}$.

The notion of a presheaf generalises essentially the following construction from set-theoretic topology. First of all, we discuss a set-theoretic examples without referring to topology. Consider an indexed family of disjoint sets:
\begin{center}
$\mathcal{A} = \{ A_i \: | \: i \in I \}$.
\end{center}
We can associate an obvious map $p : A \to I$ since for every $x \in \mathcal{A}$ there is a unique $i \in I$ such that $x \in A_i$. Take

\begin{center}
$p^{-1}(\{ i \}) = \{ x \: | \: p(x) = i\} = A_i$
\end{center}

Such $p^{-1}(\{ i \})$ is called the \emph{fibre} over $i$, the whole structure is a bundle of sets over the base space $I$, $\mathcal{A}$ is the stalk space (l'espace etale) of the bundle. More generally, we can extract the bundle from every map $p : A \to I$

A morphism of bundles $(A, I)$ and $(B, I)$ is a commutative triangle of the following form:

\xymatrix{
&&&&&& A \ar[rr]^f \ar[dr]_{p_1} && B \ar[dl]^{p_2}\\
&&&&&&& I
}

Topologically, a sheaf is a version of bundles for topological spaces. Let $\mathcal{I} = (I, \theta)$ be a topological space. A sheaf is a tuple $(\mathcal{A}, p)$, where $\mathcal{A}$ is a topological space and $p : A \to I$ is a continuous map, which is also a local homeomorphism, that is, every $x \in \mathcal{A}$ has an open neighbourhood, which mapped homeomorphically by $p$ onto $p(U)$ and $p(U)$ is open in $I$. The category of all sheaves of $I$ is sometimes called a spatial topos.

We can extract a presheaf from a sheaf $(A, f)$ as a contravariant functor $F_f : \theta \to {\bf Set}$ as
\begin{center}
$F_f(V) = \{ s : V \to A \: | \: \text{$s$ is continuous and $f \circ s = V \hookrightarrow I $}\}$
\end{center}

The category of presheafs over $I$, denoted as ${\bf PsC}(I)$, consists of presheafs as objects and natural transformations $\tau : F \Rightarrow G$, that is, a collection of functions $\tau_U : F(U) \to G(U)$ making this square commute whenever $U \subseteq V$

\xymatrix{
&&&&& F(V) \ar[d]_{F^V_U} \ar[rr]^{\tau_V} && G(V) \ar[d]^{G^V_U} \\
&&&&& F(U) \ar[rr]_{\tau_U} && G(U)
}

It is clear that ${\bf PsC}(I)$ is equivalent to ${\bf Set}^{\theta^{Op}}$.

Let $X$ be an index set and $V$ an open set, an \emph{open cover} of $V$ is a collection of sets $\{ V_x\}_{x \in X}$ such that
\begin{center}
$V = \bigcup \limits_{x \in X} V_x$
\end{center}

Intuitively, a sheaf is a presheaf that preserves open covers.

A \emph{sheaf} is a presheaf $F$ satisfying the following two extra-principles. Let $V$ be an open set and $\{V_x
\}_{x \in X}$ an open cover, then:
\begin{enumerate}
\item Let $s, t \in F(V)$ be sections such that such that $s|_{V_x} = t|_{V_x}$ for $x \in X$, then $s = t$.
\item Let $\{ s_x \in F(V_x) \}_{x \in X}$ be a family of sections. If for all $x, y \in X$ we have $s_x|_{V_x \cap V_y} = s_y|_{V_x \cap V_y}$, then there exists a section $s \in F(V)$ such that $s|_{V_x} = s_x$ for all $x \in X$.
\end{enumerate}
Equivalently, we can reformulate the latter as that $F(V) = \varprojlim_{x \in X} F(V_x)$.
The category ${\bf Sh}(I)$ is a category of sheaves over $I$.

\subsection{Grothendieck topos}

The notion of a Grothendieck topos generalises the aforementioned topological constructions. We start with the definition of a site.

Let $\mathcal{C}$ be a locally small category. A \emph{pretopology} on $\mathcal{C}$ is an assignment of each $A \in {\bf Ob}(\mathcal{C})$ of a collection of arrows $\operatorname{Cov}(A)$ (covers of $A$, or covering sieves) with the following properties:
\begin{enumerate}
\item $\{ id_A : A \to A \} \in \operatorname{Cov}(A)$
\item If $\{ f_x : A_x \to A \: | \: x \in X \} \in \operatorname{Cov}(A)$ and for each $x \in X$ we have an $a_x$-cover
\begin{center}
$\{ f_y^x : A_y^x \to A_x \: | \: y \in Y_x \} \in \operatorname{Cov}(A_x)$
\end{center}
then
\begin{center}
$\{ f_x \circ f_y^x : A^x_y \to A \: | \: x \in X, y \in Y_x\} \in  \operatorname{Cov}(A)$
\end{center}
\item If $\{ f_x : A_x \to A \: | \: x \in X \} \in \operatorname{Cov}(A)$ and $g : B \to A$ and assume that for each $x \in X$ the pullback of $f_x$ along $g$ exists:

\xymatrix{
&&&&& B \times_A A_x \ar[r] \ar[d]_{g_x} & A_x \ar[d]^{f_x}\\
&&&&& B \ar[r]_{g} & A
}
then $\{ g_x : B \times_A A_x \to B \:| \: x \in X\} \in \operatorname{Cov}(B)$
\end{enumerate}

A \emph{site} is the pair $(\mathcal{C}, \operatorname{Cov})$ consisting of a category and a pretopology on it.

A Grothendieck topos is a site with extra-conditions that generalise the axioms of topological sheaves in terms of a pretopology. A presheaf of sets over a category $\mathcal{C}$ is a contravariant functor $F : \mathcal{C} \to {\bf Set}$

Let $\operatorname{Cov}$ be a pretopology on a category $\mathcal{C}$ and $\{ f_x : A_x \to A \: | \: x \in X \} \in \operatorname{Cov}(A)$. Let $x, y \in X$ and we have the pullback of $f_x$ and $f_y$

\xymatrix{
&&&&& A_x \times_A A_y \ar[r] \ar[d] & A_y \ar[d]^{f_y}\\
&&&&& A_x \ar[r]_{f_x} & A
}

If $F$ is a presheaf over $\mathcal{C}$, then we have arrows $F^x_y : F(A_x) \to F(A_x \times_A A_y)$ and $F^y_x : F(A_y) \to F(A_x \times_A A_y)$. Denote $F_x$ as the arrow $F(f_x) : F(A) \to F(A_x)$.

A presheaf $F$ is a sheaf, if for any cover $\{ f_x : A_x \to A \: | \: x \in X \} \in \operatorname{Cov}(A)$, then for all $x, y \in X$ such that for all $s_x \in F(A_x)$ and $s_y \in F(A_y)$ such that $F^x_y(s_x) = F^y_x(y)$, then there exists a unique $s \in F(A)$ such that $F_x(s) = s_x$ for $x \in X$.

${\bf Sh}(\operatorname{Cov})$ is the category of sheaves of the site $(\mathcal{C}, \operatorname{Cov})$. A Grothendieck topos is a category of sheaves of some site up to categorical equivalence.

\section{Locales and cover systems}

\section{Kripke-Joyal semantics and quantifiers via adjoint functors}

\section{Internal logic}

\section{Geometric morphisms}

\section{Geometric logic}

\section{Kripke-Joyal semantics for quantales and non-commutative geometric theories}

\end{document}
